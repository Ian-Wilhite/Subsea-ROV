\documentclass[11pt]{article}
\usepackage[margin=1in]{geometry}
\usepackage{amsmath,amssymb,mathtools,bm}
\usepackage{siunitx}
\usepackage[hidelinks]{hyperref}
\usepackage{graphicx}
\usepackage{booktabs}
\usepackage{enumitem}
\usepackage{tocloft}
\setlength{\cftbeforesecskip}{4pt}

% --- Notation & macros ---
\newcommand{\vect}[1]{\bm{#1}}
\newcommand{\mat}[1]{\bm{#1}}
\newcommand{\R}{\mathbb{R}}
\newcommand{\x}{\vect{x}}
\newcommand{\F}{\vect{F}}
\newcommand{\Tau}{\vect{\tau}}
\newcommand{\e}{\mathrm{e}}
\newcommand{\ihat}{\hat{\imath}}
\newcommand{\jhat}{\hat{\jmath}}
\newcommand{\khat}{\hat{k}}
\newcommand{\skew}[1]{\left[#1\right]_\times}
\newcommand{\T}{^\top}

% Trig shorthands for compact matrices
\newcommand{\ct}{c_\theta}
\newcommand{\st}{s_\theta}
\newcommand{\cf}{c_\phi}
\newcommand{\sf}{s_\phi}
\newcommand{\cg}{c_\gamma}
\newcommand{\sg}{s_\gamma}

% --- Title ---
\title{Thruster Placement, Transformations, and Differentiable Evaluation for 6-DoF Wrench Authority}
\author{Ian Wilhite}
\date{\today}

\begin{document}
\maketitle

\section{Introduction}
We consider $N$ body-fixed thrusters mounted on or near a reference sphere of radius $\rho$.
Each thruster produces a force along its own local axis, and we seek to (a) map these forces to a global
wrench about the origin and (b) optimize placement/orientation for strong, well-conditioned 6-DoF authority.
This document standardizes frames, derives the required transformations, builds the allocation matrix,
and proposes a differentiable evaluation function for gradient-based design.

\begin{figure}[h!]
    \centering
    \includegraphics[width=0.5\linewidth]{Resources/Spherical ROV Force diagram(1).png}
    \caption{non-Gimbaled Thruster ROV Convention}
    \label{fig:placeholder}
\end{figure}

\section{Frames, Coordinates, and Conventions}
\paragraph{Base frame.} The global Cartesian frame $\{B\}$ has orthonormal axes $(x,y,z)$ about the origin.
\paragraph{Spherical coordinates.} We adopt $(\rho,\theta,\phi)$ with azimuth $\theta$ about $+z$ from $+x$,
and polar angle $\phi$ measured from $+z$ (north pole).
The radial unit vector and tangent basis at $(\theta,\phi)$ are
\begin{equation}
\vect e_r=
\begin{bmatrix}\sf\ct\\ \sf\st\\ \cf\end{bmatrix},\quad
\vect e_\theta=
\begin{bmatrix}-\st\\ \ct\\ 0\end{bmatrix},\quad
\vect e_\phi=
\begin{bmatrix}\cf\ct\\ \cf\st\\ -\sf\end{bmatrix},
\qquad
(\vect e_\theta\times\vect e_\phi=\vect e_r).
\end{equation}

\section{Cartesian $\rightarrow$ Spherical: Coordinates and Basis}
\paragraph{Point coordinates (nonlinear).}
For a Cartesian point $\vect p_B=(x,y,z)\T$,
\begin{equation}
\rho=\sqrt{x^2+y^2+z^2},\qquad \theta=\operatorname{atan2}(y,x),\qquad \phi=\arccos\!\left(\frac{z}{\rho}\right)\ (\rho>0).
\end{equation}
\paragraph{Basis change for vectors (linear).}
At $(\theta,\phi)$,
\begin{equation}
\label{eq:basis-c2s}
\begin{bmatrix}F_r\\ F_\theta\\ F_\phi\end{bmatrix}
=
\underbrace{\begin{bmatrix}
\sf\ct & \sf\st & \cf\\
-\st & \ct & 0\\
\cf\ct & \cf\st & -\sf
\end{bmatrix}}_{\displaystyle \mat C_{B\to\text{sph}}(\theta,\phi)}
\begin{bmatrix}F_x\\F_y\\F_z\end{bmatrix},\qquad
\begin{bmatrix}F_x\\F_y\\F_z\end{bmatrix}
=
\underbrace{\begin{bmatrix}\vect e_r & \vect e_\theta & \vect e_\phi\end{bmatrix}}_{\displaystyle \mat C_{\text{sph}\to B}}
\begin{bmatrix}F_r\\F_\theta\\F_\phi\end{bmatrix}.
\end{equation}

\section{Surface and Thruster Transforms}
Define the \emph{surface frame} $\{S\}$ at position $\rho\,\vect e_r$ with axes
$(x_s,y_s,z_s)=(\vect e_\theta,\vect e_\phi,\vect e_r)$.
The rotation and translation from $\{B\}$ to $\{S\}$ are
\begin{equation}
\label{eq:RBS}
\mat R_{BS}=\begin{bmatrix}\vect e_\theta & \vect e_\phi & \vect e_r\end{bmatrix}
=
\begin{bmatrix}
-\st & \cf\ct & \sf\ct\\
\ \ct & \cf\st & \sf\st\\
0 & -\sf & \cf
\end{bmatrix},\qquad
\vect p_{BS}=\rho\,\vect e_r.
\end{equation}
The homogeneous transform is
\begin{equation}
\mat T_{BS}=
\begin{bmatrix}
\mat R_{BS} & \vect p_{BS}\\
\mathbf 0^\top & 1
\end{bmatrix}.
\end{equation}
Let the \emph{thruster frame} $\{T\}$ be obtained by a rotation $\gamma$ about $z_s=\vect e_r$
within the tangent plane, then a standoff $d$ along $+z_s$:
\begin{equation}
\label{eq:RST}
\mat R_{ST}(\gamma)=
\begin{bmatrix}
\cg & -\sg & 0\\
\sg & \ \cg & 0\\
0 & 0 & 1
\end{bmatrix},\qquad
\vect p_{ST}=d\begin{bmatrix}0\\0\\1\end{bmatrix}_{\{S\}}.
\end{equation}
The cumulative transform is
\begin{equation}
\mat T_{BT}=\mat T_{BS}\,\mat T_{ST},\qquad
\mat R_{BT}=\mat R_{BS}\mat R_{ST},\qquad
\vect p_{BT}=\vect p_{BS}+\mat R_{BS}\vect p_{ST}=(\rho+d)\,\vect e_r.
\end{equation}

\section{Force and Wrench Mapping}
A pure force is rotated but not translated:
\begin{equation}
\label{eq:MF}
\vect F_B=\underbrace{\mat R_{BT}}_{\displaystyle \mat M_F(\theta,\phi,\gamma)}\,\vect F_T.
\end{equation}
Carrying out the product $\mat R_{BS}\mat R_{ST}$ yields
\begin{equation}
\mat M_F(\theta,\phi,\gamma)=
\begin{bmatrix}
-\cg\,\st+\sg\,\cf\,\ct & \ \sg\,\st+\cg\,\cf\,\ct & \ \sf\,\ct\\
\ \cg\,\ct+\sg\,\cf\,\st & -\sg\,\ct+\cg\,\cf\,\st & \ \sf\,\st\\
-\sg\,\sf & -\cg\,\sf & \cf
\end{bmatrix}.
\end{equation}
If the thruster produces a wrench $(\vect F_T,\vect \tau_T)$ at $\{T\}$, the base wrench is
\begin{equation}
\begin{bmatrix}\vect F_B\\ \vect \tau_B\end{bmatrix}
=
\underbrace{\begin{bmatrix}
\mat R_{BT} & \mathbf 0\\
\skew{\vect p_{BT}}\mat R_{BT} & \mat R_{BT}
\end{bmatrix}}_{\displaystyle \mat A^\ast(\rho,\theta,\phi,\gamma,d)}
\begin{bmatrix}\vect F_T\\ \vect \tau_T\end{bmatrix},
\end{equation}
where $\skew{\vect p}$ is the cross-product matrix.

\section{Thruster Wrench Basis ($6\times N$ Allocation)}
For thruster $i$ with parameters $(\rho_i,\theta_i,\phi_i,\gamma_i,d_i)$, position
$\vect r_i=(\rho_i+d_i)\,\vect e_{r_i}$ and unit tangential direction
$\hat{\vect f}_i=\cos\gamma_i\,\vect e_{\theta_i}+\sin\gamma_i\,\vect e_{\phi_i}$, define the unit wrench
\begin{equation}
\vect b_i=\begin{bmatrix}\hat{\vect f}_i\\ \vect r_i\times \hat{\vect f}_i\end{bmatrix}\in\R^{6}.
\end{equation}
Stacking columns yields the allocation matrix
\begin{equation}
\label{eq:B}
\mat B=\begin{bmatrix}\vect b_1 & \vect b_2 & \cdots & \vect b_N\end{bmatrix}\in\R^{6\times N},
\qquad
\begin{bmatrix}\vect F\\ \vect \tau\end{bmatrix}=\mat B\,\vect u,
\end{equation}
where $\vect u\in\R^N$ are thrust magnitudes (signed if bidirectional).

\section{Differentiable Evaluation of the Basis}
For forces (N) and torques (N\,m) we adopt a balancing length $R_{\text{char}}>0$ and define
\begin{equation}
\mat W=\mathrm{diag}\!\Big(1,1,1,\tfrac{1}{R_{\text{char}}},\tfrac{1}{R_{\text{char}}},\tfrac{1}{R_{\text{char}}}\Big),
\qquad
\widetilde{\mat B}=\mat W\,\mat B,\qquad
\mat M=\widetilde{\mat B}\,\widetilde{\mat B}\T.
\end{equation}
With a small $\varepsilon>0$, we combine three smooth terms into a maximization objective:
\begin{align}
J_{\text{vol}}(\mat B) &= \log\det\!\big(\mat M+\varepsilon \mat I_6\big),\\
P_{\text{iso}}(\mat B) &= \left\|\mat M-\tfrac{\mathrm{tr}(\mat M)}{6}\mat I_6\right\|_F^2,\\
P_{\text{coh}}(\mat B) &= \sum_{i\neq j}\left(\frac{\widetilde{\vect b}_i\T \widetilde{\vect b}_j}{\|\widetilde{\vect b}_i\|\,\|\widetilde{\vect b}_j\|}\right)^{\!2},
\end{align}
and the combined score
\begin{equation}
\label{eq:objective}
\boxed{~J(\mat B)=\alpha\,J_{\text{vol}}(\mat B)-\beta\,P_{\text{iso}}(\mat B)-\gamma\,P_{\text{coh}}(\mat B),~}
\end{equation}
with weights $\alpha,\beta,\gamma>0$. The gradient of $J_{\text{vol}}$ admits the closed form
\begin{equation}
\frac{\partial J_{\text{vol}}}{\partial \widetilde{\mat B}}=2\big(\mat M+\varepsilon \mat I_6\big)^{-1}\widetilde{\mat B},
\qquad
\frac{\partial J_{\text{vol}}}{\partial \mat B}=\mat W^\top\frac{\partial J_{\text{vol}}}{\partial \widetilde{\mat B}}.
\end{equation}

\section{Optimization Statement and Practical Constraints}
We optimize angular parameters $\{\phi_i,\theta_i,\gamma_i\}_{i=1}^N$ (and optionally $d_i$) subject to
bounds and collision/clearance constraints:
\begin{align}
\max_{\{\phi_i,\theta_i,\gamma_i\}} \quad & J\!\left(\mat B(\{\phi_i,\theta_i,\gamma_i\})\right)\\
\text{s.t.}\quad & \phi_i\in[0,\pi],\ \theta_i\in[0,2\pi),\ \gamma_i\in[0,2\pi),\\
& d_i\in[d_{\min},d_{\max}],\ \text{clearances, FOV, wiring, and mechanical limits.}
\end{align}
Choose $R_{\text{char}}$ as a representative arm length, e.g. mean $(\rho_i+d_i)$.

\section{Conclusion}
This narrative consolidates geometry, transforms, wrench mapping, the allocation matrix,
and a differentiable objective suitable for gradient-based thruster placement.
It is intended to be self-contained and implementation-ready.


\end{document}
