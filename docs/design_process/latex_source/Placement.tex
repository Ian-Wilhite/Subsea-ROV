\documentclass[11pt]{article}
\usepackage[margin=1in]{geometry}
\usepackage{amsmath,amssymb,mathtools,bm}
\usepackage{siunitx}
\usepackage[hidelinks]{hyperref}
\usepackage{graphicx}
\usepackage{booktabs}
\usepackage{enumitem}
\usepackage{tocloft}
\setlength{\cftbeforesecskip}{4pt}

% --- Notation & macros ---
\newcommand{\vect}[1]{\bm{#1}}
\newcommand{\mat}[1]{\bm{#1}}
\newcommand{\R}{\mathbb{R}}
\newcommand{\x}{\vect{x}}
\newcommand{\F}{\vect{F}}
\newcommand{\Tau}{\vect{\tau}}
\newcommand{\e}{\mathrm{e}}
\newcommand{\ihat}{\hat{\imath}}
\newcommand{\jhat}{\hat{\jmath}}
\newcommand{\khat}{\hat{k}}
\newcommand{\skew}[1]{\left[#1\right]_\times}
\newcommand{\T}{^\top}

% Trig shorthands for compact matrices
\newcommand{\ct}{c_\theta}
\newcommand{\st}{s_\theta}
\newcommand{\cf}{c_\phi}
\newcommand{\sf}{s_\phi}
\newcommand{\cg}{c_\gamma}
\newcommand{\sg}{s_\gamma}

% --- Title ---
\title{Thruster Placement Optimization}
\author{Ian Wilhite}
\date{\today}

\begin{document}
\maketitle

\section{Introduction}
We consider $N$ body-fixed thrusters mounted on or near a reference sphere of radius $\rho$.
Each thruster produces a force along its own local axis, and we seek to (a) map these forces to a global
wrench about the origin and (b) optimize placement/orientation for strong, well-conditioned 6-DoF authority.
This document standardizes frames, derives the required transformations, builds the allocation matrix,
and proposes a differentiable evaluation function for gradient-based design.

While a minimum of six thrusters are required for full 6-DoF control, this system utilizes eight. The two additional thrusters provide redundancy, enchancing the vehicle's fault tolerance and allowing it to maintain control with failure. This redundancy was evaluated using a 'leave-one-out' analysis, which accounts for a worst-case single-point failure, aiding the evaluation of not only complete control authority, but designing redundancy into the system.

\begin{figure}[h!]
    \centering
    \includegraphics[width=0.5\linewidth]{Resources/Spherical ROV Force diagram(1).png}
    \caption{non-Gimbaled Thruster ROV Convention}
    \label{fig:placeholder}
\end{figure}

% elbow plot to pick 8 thrusters

% evalution function (revisions)
% s


% stochastic gradient ascent 





\begin{figure}
    \centering
    \includegraphics[width=0.5\linewidth]{Resources/eval_comp_vs_n.png}
    \caption{Evaluation Components}
    \label{fig:placeholder}
\end{figure}

\section{Optimization Process}

The problem of thruster placement optimization on the ROV, as defined in this document, is to find a thruster configuration that provides well-conditioned 6-DoF control with wrench bias under single-point failures.

Here is a narrative of the optimization process:

\subsection{Stage 1: Initial Exploration (\texttt{optimizer.py})}

The process follows standard stochastic gradient descent methods. An evalaution function was developed which evaluated 

\subsection{Stage 2: Mathematical Formulation and Core Metrics (\texttt{optimizer2.py})}

This script marks the true beginning of the rigorous optimization process. It established the core mathematical framework by defining:

\begin{itemize}
    \item \textbf{The Allocation Matrix (A):} The central element of the problem. This 6xN matrix maps the forces of the N individual thrusters to the resulting 6-DoF wrench (3 forces + 3 torques) on the ROV's body.
    \item \textbf{Wrench Metrics:} To evaluate the quality of a given thruster layout, a set of metrics based on the Singular Value Decomposition (SVD) of the allocation matrix was implemented:
    \begin{itemize}
        \item \texttt{sigma\_min}: The smallest singular value, representing the maneuverability in the weakest direction. Maximizing this is key to ensuring the ROV has no "bad" directions.
        \item \texttt{manipulability}: The product of the singular values. A measure of the total volume of the achievable wrench space.
        \item \texttt{cond} (Condition Number): The ratio of the largest to the smallest singular value. A value close to 1 indicates that the ROV is equally agile in all directions (isotropy).
    \end{itemize}
    \item \textbf{Objective Function (J):} A scalar objective function \texttt{J} was created to be minimized. It was formulated to simultaneously maximize \texttt{sigma\_min} and \texttt{manipulability} while minimizing the \texttt{condition number}.
    \item \textbf{Random Search:} A \texttt{quick\_random\_search} function was implemented to sample thousands of random thruster configurations to find a promising starting point for a more refined search.
\end{itemize}

\subsection{Stage 3: Gradient Descent and Visualization (\texttt{optimizer3.py} and \texttt{optimizer4.py})}

Building on the mathematical framework of the previous stage, these scripts (which are identical) introduced an iterative optimization method:

\begin{itemize}
    \item \textbf{Gradient Descent:} A gradient descent loop was implemented to refine the thruster parameters. The gradient of the complex objective function was calculated numerically using a finite-difference method. In each iteration, the parameters (thruster angles) are adjusted to "descend" towards a better objective score.
    \item \textbf{3D Visualization:} The script added \texttt{matplotlib} visualizations to plot the thruster positions and orientations on a 3D sphere, providing immediate visual feedback on the optimizer's progress.
    \item \textbf{Effectiveness Evaluation:} A critical function, \texttt{evaluate\_effectiveness}, was added. It uses linear programming to calculate the maximum achievable force/torque in each of the 12 cardinal directions (e.g., +Fx, -Fx, +Fy, etc.), taking into account realistic asymmetric thrust limits. This provides a concrete measure of the final design's real-world performance.
\end{itemize}

\subsection{Stage 4: Structured, Goal-Oriented Optimization (\texttt{optimizer5.py})}

This final script represents a significant leap in sophistication by introducing structure and user-defined goals:

\begin{itemize}
    \item \textbf{Reduced Parameter Space:} Instead of optimizing 24 independent parameters for 8 thrusters, the problem was constrained. The thrusters were grouped into 4 pairs, with the pair centers fixed in a symmetric tetrahedral arrangement. This reduced the optimization space to just 12 parameters (\texttt{spacing}, \texttt{tilt}, and \texttt{rotation} for each pair), making the search more efficient.
    \item \textbf{Goal-Oriented Objective Function:} The objective function was enhanced to be goal-driven. A \texttt{w\_target} (target wrench) vector was introduced, allowing the user to specify desired performance characteristics (e.g., "more forward thrust than vertical thrust"). The objective function now includes a penalty for any deviation from this target, guiding the optimizer to a solution that is not only well-conditioned but also tailored to the specific mission requirements of the ROV.
    \item \textbf{Advanced Diagnostics:} The script includes tools to plot the convergence of the objective function and to quantitatively compare the final achieved wrench capabilities against the desired target wrench, providing a clear report on the success of the optimization.
\end{itemize}

In essence, the process evolved from a simple idea to a powerful, goal-driven optimization framework that leverages advanced mathematical concepts and numerical methods to design a bespoke, high-performance thruster configuration for the ROV.

\section{Conclusion}
This narrative consolidates geometry, transforms, wrench mapping, the allocation matrix,
and a differentiable objective suitable for gradient-based thruster placement.
It is intended to be self-contained and implementation-ready.


\end{document}

